\documentclass[10pt]{beamer}
\usetheme{journeys_econark}

% Math packages
\usepackage{amsmath}
\usepackage{amsfonts}
\usepackage{amssymb}
\usepackage{mathtools}
\usepackage{bm}  % For bold math symbols

% Graphics and colors
\usepackage{graphicx}
\usepackage{xcolor}
\usepackage{tikz}
\usepackage{dsfont}
\usepackage{bbm}

% Bibliography
\usepackage[style=authoryear,backend=biber]{biblatex}

%----- CUSTOM MACROS -----------------------------------------------
% Expectations and spaces
\newcommand{\E}{\mathbb{E}}
\newcommand{\R}{\mathbb{R}}
\newcommand{\Z}{\mathcal{Z}}
\newcommand{\F}{\mathcal{F}}
\newcommand{\C}{\mathcal{C}}

% Probability notation
\newcommand{\Prob}{\mathbb{P}}
\renewcommand{\P}{\mathbb{P}}

% Operators
\newcommand{\T}{\mathbf{T}}

% Functions (in Roman font following Stachurski style)
\newcommand{\ubar}{\underline{u}}
\newcommand{\cstar}{c^*}
\newcommand{\wbar}{\bar{w}}

% MPC and saving
\newcommand{\MPC}{\text{MPC}}
\newcommand{\AMPC}{\overline{\text{MPC}}}

% Risk aversion
\newcommand{\RA}{\gamma}

%----- THEOREM ENVIRONMENTS ----------------------------------------
% No automatic numbering - we'll use manual numbers from the paper
\let\theorem\relax
\newtheorem*{theorem}{Theorem}
\newtheorem*{assumption}{Assumption}
\newtheorem*{claim}{Claim}
\newtheorem*{proposition}{Proposition}
\let\definition\relax
\newtheorem*{definition}{Definition}
\newtheorem*{remark}{Remark}

%----- METADATA ----------------------------------------------------
\title{A Theory of Saving under Risk Preference Dynamics}
\subtitle{Ma, Song \& Toda (2025) - Reading Group Presentation}
\author{Presented by: Akshay Shanker}
\institute{University of New South Wales}
\date{Tv=v, 19 November 2025}

%=====================================================================
\begin{document}

%----- TITLE SLIDE ---------------------------------------------------
\begin{frame}
  \titlepage
\end{frame}

%----- PAPER OVERVIEW ------------------------------------------------
\begin{frame}
  \frametitle{Paper Overview}

  \structure{Authors:} Qingyin Ma, Xinxi Song, Alexis Akira Toda (2025)

  \vspace{0.5em}

  \structure{Central Questions:}
  \begin{itemize}
    \item What explains the high saving rates among wealthy households?
    \item How can macroeconomic models generate empirically realistic wealth distributions?
  \end{itemize}

  \vspace{0.5em}

  \structure{Key Innovation:} Time-varying risk aversion.

  \vspace{0.5em}

  \structure{Main Finding:}
  \begin{itemize}
    \item Zero asymptotic MPCs arise naturally.
    \item No need for complex return processes.
  \end{itemize}
\end{frame}

%----- OUTLINE -------------------------------------------------------
\begin{frame}
  \frametitle{Outline}

  \tableofcontents
\end{frame}

%=====================================================================
\section{Motivation}

%----- EMPIRICAL PUZZLE ----------------------------------------------
\begin{frame}
  \frametitle{The Empirical Puzzle}

  \structure{Stylized Facts}
  \begin{itemize}
    \item Wealthy households have \alert{substantially higher saving rates}.
    \item Top 1\% exhibit \alert{markedly lower MPC} than median households.
    \item This pattern persists across countries and time periods.
  \end{itemize}

  \vspace{1em}

  \structure{Theoretical Challenge}

  Existing models require \alert{restrictive assumptions} to yield zero asymptotic MPCs:
  \begin{itemize}
    \item Ma \& Toda (2021): Need stringent conditions on return risk.
    \item Benhabib et al. (2015): Require specific capital income risk structure.
    \item Carroll and Shanker (2025): $(\frac{R}{\beta})^{1/\gamma} \geq R$.
  \end{itemize}
\end{frame}


\begin{frame}
  \frametitle{Relevance of Studying Asymptotic MPCs}

  The literature lacks consensus on the mechanisms underlying high saving rates among affluent households. 
  %
  \vspace{1em}

  \structure{Theoretical Importance:}
  \begin{itemize}
    \item Requires a unified consumption theory explaining behavior across the wealth distribution.
    \item Critical for policy design with heterogeneous agents and impulse response analysis.
  \end{itemize}

\structure{Potential Limitations:}
\begin{itemize}
  \item Unclear whether the core issue concerns \alert{MPCs}, \alert{distributional tails}, \alert{risk perceptions}, or \alert{market frictions}.
  \item The inability of standard models to generate zero asymptotic MPCs may not constitute sufficient grounds for model rejection.
\end{itemize}
\end{frame}


%----- KEY INNOVATION ------------------------------------------------
\begin{frame}
  \frametitle{Paper's Key Contribution}

  Introduces stochastic risk aversion into the standard \alert{income fluctuation problem}. 
  %
  \vspace{0.5em}

  \structure{Methodological Innovations:}
  \begin{itemize}
    \item Risk preferences exhibit state-dependent and temporal variation.
    \item Incorporates documented empirical evidence of preference heterogeneity.
    \item Eliminates reliance on complex return processes or restrictive parameter conditions.
  \end{itemize}

  \vspace{0.5em}

  \begin{alertblock}{Principal Finding}
    Zero asymptotic MPCs emerge endogenously when agents face potential transitions to lower risk aversion states.
  \end{alertblock}


\end{frame}

%=====================================================================
\section{Model Setup}

%----- OPTIMAL SAVINGS PROBLEM --------------------------------------
\begin{frame}
  \frametitle{The Optimal Savings Problem}

  \structure{Agent's Dynamic Problem}
  \begin{align*}
    \max_{\{c_t, w_t\}} \quad & \E_0 \left[ \sum_{t=0}^{\infty} \left(\prod_{i=0}^{t} \beta_i\right) u(c_t, Z_t) \right] \\[0.3em]
    \text{s.t.} \quad & w_{t+1} = R_{t+1}(w_t - c_t) + Y_{t+1} \\
    & 0 \leq c_t \leq w_t
  \end{align*}

  \structure{Key State Variables:}
  \begin{itemize}
    \item $Z_t$: Markov chain with \alert{preference shocks}.
    \item $\beta_t$: Stochastic discount factor.
    \item $R_t$: Stochastic returns.
    \item $Y_t$: Non-financial income.
  \end{itemize}

  \vspace{0.3em}
  \structure{Novel feature:} $Z_t$ affects risk aversion $\gamma(Z_t)$ directly.
\end{frame}

%----- PREFERENCE SPECIFICATION -------------------------------------
\begin{frame}
  \frametitle{State-Dependent Risk Aversion}

  \structure{Utility specification}
  \begin{equation*}
    u(c, z) = \begin{cases}
      \frac{c^{1-\gamma(z)}}{1-\gamma(z)} & \text{if } \gamma(z) > 0, \gamma(z) \neq 1 \\
      \log c & \text{if } \gamma(z) = 1
    \end{cases}
  \end{equation*}

  \vspace{3em}

  Key feature: \alert{$\gamma(z)$ varies with state $z$}.

  \vspace{1em}

  \begin{itemize}
    \item State decomposition: $Z_t = (\bar{Z}_t, \tilde{Z}_t)$.
    \item Risk aversion driven by $\bar{Z}_t \in \{\bar{z}_1, \ldots, \bar{z}_N\}$.
    \item Ordering: $0 < \gamma(\bar{z}_1) < \cdots < \gamma(\bar{z}_N)$.
    \item Transition matrix: $\bar{P} = (\bar{p}_{ij})_{1 \leq i,j \leq N}$.
  \end{itemize}
\end{frame}

%----- EULER EQUATION ------------------------------------------------
\begin{frame}
  \frametitle{Optimality Conditions}

  \structure{Euler Equation}

  The optimal consumption function $c^*(w,z)$ satisfies:

  \begin{equation*}
    u'(c^*(w,z), z) = \max\left\{ \E_z[\hat{\beta}\hat{R}u'(c^*(\hat{w}, \hat{Z}), \hat{Z})], u'(w,z) \right\}
  \end{equation*}
  where $\hat{w} = \hat{R}(w - c^*(w,z)) + \hat{Y}$

  \vspace{1em}

\end{frame}

%----- KEY ASSUMPTION ------------------------------------------------
\begin{frame}
  \frametitle{Key Assumption for Main Results}

  \begin{assumption}[2.2: Returns and Discounting]
    \begin{enumerate}
      \item For all $z \in Z$:
      \begin{itemize}
        \item $\E_z[u_c(\hat{Y}, \hat{Z})] < \infty$
        \item $\E_z[\hat{\beta}\hat{R} \cdot u_c(\hat{Y}, \hat{Z})] < \infty$
      \end{itemize}

      \item \structure{Spectral radius condition:} $r(K(1)) < 1$

      where the matrix $K(\theta)$ is defined by:
      \begin{equation*}
        K_{zz'}(\theta) = P(z,z') \int \beta(z,z',\varepsilon) R(z,z',\varepsilon)^{\theta} \pi(d\varepsilon)
      \end{equation*}
    \end{enumerate}
  \end{assumption}

  \begin{alertblock}{Economic Interpretation}
    The condition $r(K(1)) < 1$ ensures convergence of wealth in present value terms\\
    Generalizes the standard condition $\beta R < 1$
  \end{alertblock}
\end{frame}

%=====================================================================
\section{Main Results}

%----- THEOREM 2.1: EXISTENCE ----------------------------------------
\begin{frame}
  \frametitle{Existence and Uniqueness}

  \begin{theorem}[2.1: Existence and Uniqueness]
    If Assumptions 2.1 and 2.2 hold, then:
    \begin{enumerate}
      \item The time iteration operator $T:\mathcal{C}\to\mathcal{C}$ has a unique fixed point $c^*$.
      \item For any $c \in \mathcal{C}$, $\rho(T^k c, c^*) \to 0$ as $k\to\infty$.
    \end{enumerate}
  \end{theorem}

  \vspace{0.5em}

  \structure{Approach:}
  \begin{itemize}
    \item Work in space $\mathcal{C}$ of continuous consumption functions.
    \item Use marginal utility metric: $\rho(c_1, c_2) := \sup_{(w,z)} |u'(c_1(w,z), z) - u'(c_2(w,z), z)|$.
    \item Time iteration operator contracts in this metric.
  \end{itemize}
\end{frame}

%----- THEOREM 3.2: ZERO MPC ----------------------------------------
\begin{frame}
  \frametitle{Zero Asymptotic MPCs}

  \begin{theorem}[3.2: Positive Transition Matrix]
    If every entry of $\bar{P}$ is strictly positive and $\Pr_{z,z'}(\hat{\beta}\hat{R} > 0) > 0$ for all $(z,z') \in Z^2$, then
      \begin{equation*}
        \lim_{w \to \infty} \frac{c^*(w, z_{ij})}{w} = 0 \quad \text{for all states with } i \neq 1.
      \end{equation*}
  \end{theorem}

  \vspace{0.5em}

  \begin{alertblock}{Key Insight (generalized from Theorem 3.2)}
    Under a fully mixing risk aversion Markov chain, zero asymptotic MPCs obtain at all non-minimal risk aversion states.
  \end{alertblock}
\end{frame}

%----- INTUITION FOR MAIN RESULT ------------------------------------
\begin{frame}
  \frametitle{Economic Intuition}

  \begin{center}
    \large What mechanism links potential risk aversion reduction to enhanced savings behavior?
  \end{center}

  \vspace{1em}

  \structure{Theoretical Mechanism}
  \begin{enumerate}
    \item Agents anticipate potential \alert{transitions} to lower risk aversion states.
    \item Reduced future risk aversion $\Rightarrow$ Increased marginal utility of future consumption.
    \item Generates \alert{persistent precautionary saving incentives}.
    \item Precautionary motive remains operative across all wealth levels.
  \end{enumerate}

  \vspace{0.5em}
\end{frame}

\begin{frame}
  \frametitle{Comparison with Existing Theory}
  
  \structure{Departure from Existing Literature:}

  \begin{itemize}
      \item Ma \& Toda (2021): Zero MPC requires knife-edge parameter restrictions.
      \item \alert{Present framework}: Zero MPC emerges as a generic equilibrium outcome.
  \end{itemize}
  \structure{Result obtains under simplified conditions:}
    \begin{itemize}
      \item Deterministic returns ($R_t \equiv R$).
      \item Absence of income uncertainty.
      \item Time-invariant discount factor.
  \end{itemize}
\end{frame}

%----- THEOREM 3.4: UPWARD TRANSITIONS ------------------------------
\begin{frame}
  \frametitle{Contrasting Result: Upward Transitions}

  \begin{theorem}[3.4: Strictly Increasing Risk Aversion]
    If risk aversion \alert{strictly increases (?)} from each state $i$ and $R$ is bounded below by $m > 0$, then
    \begin{equation*}
      \lim_{w \to \infty} \frac{c^*(w, z_{ij})}{w} = 1 \quad \text{for all } j.
    \end{equation*}
    (consumption function is \alert{nonconcave}).
  \end{theorem}

  \vspace{0.5em}

\end{frame}

\begin{frame}
  \frametitle{Upward vs Downward Transitions}

  \begin{alertblock}{Either Or}
    \begin{itemize}
      \item \structure{Theorem 3.2}: Downward risk aversion transitions $\Rightarrow$ $\bar{c} = 0$ (complete wealth retention).
      \item \structure{Theorem 3.4}: Upward risk aversion transitions $\Rightarrow$ $\bar{c} = 1$ (complete wealth depletion).
    \end{itemize}
  \end{alertblock}

  \structure{Economic Intuition:} Anticipated increases in future risk aversion induce \alert{intertemporal substitution} toward present consumption.
\end{frame}

%----- THEOREM 3.3: POSITIVE ASYMPTOTIC MPC -------------------------
\begin{frame}
  \frametitle{When MPCs Can Be Positive}

  \vspace{0.5em}

  \begin{proposition}[3.1: No Downward Transitions Required]
    If $\bar{c}(z_{ij}) > 0$ for some $i,j$, then:
    \begin{equation*}
      \bar{p}_{i1} = \cdots = \bar{p}_{i,i-1} = 0.
    \end{equation*}
    (Zero probability of moving to lower risk aversion).
  \end{proposition}
\end{frame}

%----- SUMMARY OF MAIN THEOREMS -------------------------------------
\begin{frame}
  \frametitle{Summary of Main Theorems}


  \structure{Central Implications:}
  \begin{itemize}
    \item Zero MPCs constitute a \alert{generic} outcome when risk aversion permits downward transitions.
    \item Strictly positive MPCs require \alert{restrictive} conditions that preclude transitions to lower risk aversion states.
  \end{itemize}
\end{frame}

%----- COMPARISON WITH LITERATURE -----------------------------------
\begin{frame}
  \frametitle{Comparison with Existing Literature}
  \vspace{0.8em}

  \structure{Theoretical Significance:}
  \begin{itemize}
    \item Provides a parsimonious mechanism independent of growth rate assumptions.
    \item Generates robust theoretical predictions across parameter spaces.
  \end{itemize}

  \structure{Outstanding Questions:}
  \begin{itemize}
    \item Is zero limiting MPC necessary? What \alert{specific empirical regularities} require explanation?
    \item Does the mechanism operate through \alert{risk channels} or \alert{intertemporal substitution}?
    \item The literature now contains \alert{multiple model-specific results} with disparate \alert{growth conditions}.
    \item External validity: Does preference dynamics satisfy the \underline{\alert{"Kath Day-Knight Test"}}?
  \end{itemize}
\end{frame}

\begin{frame}
  \frametitle{Kath Day-Night Test}

  \begin{center}
    Question: Why is your MPC limiting to zero?
    %
    \includegraphics[width=0.55\textwidth]{kath-kel-quiz-2.png}
  \end{center}

\end{frame}
%=====================================================================
\section{Implications}



\end{document}