\documentclass[10pt]{beamer}
\usetheme{journeys_econark}

% Math packages
\usepackage{amsmath}
\usepackage{amsfonts}
\usepackage{amssymb}
\usepackage{mathtools}
\usepackage{bm}  % For bold math symbols

% Graphics and colors
\usepackage{graphicx}
\usepackage{xcolor}
\usepackage{tikz}
\usepackage{dsfont}
\usepackage{bbm}

% Bibliography
\usepackage[style=authoryear,backend=biber]{biblatex}

%----- CUSTOM MACROS -----------------------------------------------
% Expectations and spaces
\newcommand{\E}{\mathbb{E}}
\newcommand{\R}{\mathbb{R}}
\newcommand{\Z}{\mathcal{Z}}
\newcommand{\F}{\mathcal{F}}
\newcommand{\C}{\mathcal{C}}

% Probability notation
\newcommand{\Prob}{\mathbb{P}}
\renewcommand{\P}{\mathbb{P}}

% Operators
\newcommand{\T}{\mathbf{T}}

% Functions (in Roman font following Stachurski style)
\newcommand{\ubar}{\underline{u}}
\newcommand{\cstar}{c^*}
\newcommand{\wbar}{\bar{w}}

% MPC and saving
\newcommand{\MPC}{\text{MPC}}
\newcommand{\AMPC}{\overline{\text{MPC}}}

% Risk aversion
\newcommand{\RA}{\gamma}

%----- THEOREM ENVIRONMENTS ----------------------------------------
% No automatic numbering - we'll use manual numbers from the paper
\let\theorem\relax
\newtheorem*{theorem}{Theorem}
\newtheorem*{assumption}{Assumption}
\newtheorem*{claim}{Claim}
\newtheorem*{proposition}{Proposition}
\let\definition\relax
\newtheorem*{definition}{Definition}
\newtheorem*{remark}{Remark}

%----- METADATA ----------------------------------------------------
\title{A Theory of Saving under Risk Preference Dynamics}
\subtitle{Ma, Song \& Toda (2024) - Reading Group Presentation}
\author{Presented by: Akshay Shanker}
\institute{University of New South Wales}
\date{Tv=v, 19 November 2025}

%=====================================================================
\begin{document}

%----- TITLE SLIDE ---------------------------------------------------
\begin{frame}
  \titlepage
\end{frame}

%----- PAPER OVERVIEW ------------------------------------------------
\begin{frame}
  \frametitle{Paper Overview}

  \structure{Authors:} Qingyin Ma, Xinxi Song, Alexis Akira Toda (2024)

  \vspace{0.5em}

  \structure{Central Question:}
  \begin{itemize}
    \item Why do wealthy households save so much?
    \item How to generate realistic wealth distributions in macro models?
  \end{itemize}

  \vspace{0.5em}

  \structure{Key Innovation:} Time-varying risk aversion

  \vspace{0.5em}

  \structure{Main Finding:}
  \begin{itemize}
    \item Zero asymptotic MPCs arise naturally
    \item No need for complex return processes
  \end{itemize}
\end{frame}

%----- OUTLINE -------------------------------------------------------
\begin{frame}
  \frametitle{Outline}

  \tableofcontents
\end{frame}

%=====================================================================
\section{Motivation}

%----- EMPIRICAL PUZZLE ----------------------------------------------
\begin{frame}
  \frametitle{The Empirical Puzzle}

  \structure{Stylized Facts}
  \begin{itemize}
    \item Wealthy households have \alert{substantially higher saving rates}
    \item Top 1\% exhibit \alert{markedly lower MPC} than median households
    \item This pattern persists across countries and time periods
  \end{itemize}

  \vspace{1em}

  \structure{Theoretical Challenge}

  Existing models require \alert{restrictive assumptions} to yield zero asymptotic MPCs:
  \begin{itemize}
    \item Ma \& Toda (2021): Need stringent conditions on return risk
    \item Benhabib et al. (2015): Require specific capital income risk structure
    \item Carroll and Shanker (2025): $(\frac{R}{\beta})^{1/\gamma} \geq R$
  \end{itemize}
\end{frame}


\begin{frame}
  \frametitle{Relevance of studying asymptotic MPCs}

  Overall, we seem to have a poor understanding of why the rich are thrifty. 
  %
  \vspace{1em}

  \structure{Why it matters:}
  \begin{itemize}
    \item Need a theory of consumption that explains behaviour across agent types.
    \item Matters for policy design with heterogeneous agents and generating IRFs.
  \end{itemize}

\structure{Why it (may) not matter:}
\begin{itemize}
  \item Is the question about \alert{MPCs}, \alert{tails}, \alert{risk perceptions} or \alert{frictions}?
  \item We do not really know, but saying asymptotic MPCs cannot be zero in a standard model \alert{may} not be enough motivation to reject it.
\end{itemize}
\end{frame}


%----- KEY INNOVATION ------------------------------------------------
\begin{frame}
  \frametitle{Paper's Key Contribution}

  Incorporate stochastic risk aversion in an otherwise vanilla \alert{income fluctuation} 
  \alert{problem}. 
  %
  \vspace{0.5em}

  \structure{Key Features:}
  \begin{itemize}
    \item Risk preferences vary across states and over time
    \item Captures empirical evidence of preference heterogeneity
    \item No need for complex return processes/ edge case conditions on returns and discounting
  \end{itemize}

  \vspace{0.5em}

  \begin{alertblock}{Main Result}
    Zero asymptotic MPCs arise when agents can become less risk averse in the future.
  \end{alertblock}


\end{frame}

%=====================================================================
\section{Model Setup}

%----- OPTIMAL SAVINGS PROBLEM --------------------------------------
\begin{frame}
  \frametitle{The Optimal Savings Problem}

  \structure{Agent's Dynamic Problem}
  \begin{align*}
    \max_{\{c_t, w_t\}} \quad & \E_0 \left[ \sum_{t=0}^{\infty} \left(\prod_{i=0}^{t} \beta_i\right) u(c_t, Z_t) \right] \\[0.3em]
    \text{s.t.} \quad & w_{t+1} = R_{t+1}(w_t - c_t) + Y_{t+1} \\
    & 0 \leq c_t \leq w_t
  \end{align*}

  \structure{Key State Variables:}
  \begin{itemize}
    \item $Z_t$: Markov chain with \alert{preference shocks}
    \item $\beta_t$: Stochastic discount factor
    \item $R_t$: Stochastic returns
    \item $Y_t$: Non-financial income
  \end{itemize}

  \vspace{0.3em}
  \structure{Novel feature:} $Z_t$ affects risk aversion $\gamma(Z_t)$ directly.
\end{frame}

%----- PREFERENCE SPECIFICATION -------------------------------------
\begin{frame}
  \frametitle{State-dependent risk aversion}

  \structure{Utility specification}
  \begin{equation*}
    u(c, z) = \begin{cases}
      \frac{c^{1-\gamma(z)}}{1-\gamma(z)} & \text{if } \gamma(z) > 0, \gamma(z) \neq 1 \\
      \log c & \text{if } \gamma(z) = 1
    \end{cases}
  \end{equation*}

  \vspace{3em}

  Key feature: \alert{$\gamma(z)$ varies with state $z$}.

  \vspace{1em}

  \begin{itemize}
    \item State decomposition: $Z_t = (\bar{Z}_t, \tilde{Z}_t)$
    \item Risk aversion driven by $\bar{Z}_t \in \{\bar{z}_1, \ldots, \bar{z}_N\}$.
    \item Ordering: $0 < \gamma(\bar{z}_1) < \cdots < \gamma(\bar{z}_N)$.
    \item Transition matrix: $\bar{P} = (\bar{p}_{ij})_{1 \leq i,j \leq N}$.
  \end{itemize}
\end{frame}

%----- EULER EQUATION ------------------------------------------------
\begin{frame}
  \frametitle{Optimality Conditions}

  \structure{Euler Equation}

  The optimal consumption function $c^*(w,z)$ satisfies:

  \begin{equation*}
    u'(c^*(w,z), z) = \max\left\{ \E_z[\hat{\beta}\hat{R}u'(c^*(\hat{w}, \hat{Z}), \hat{Z})], u'(w,z) \right\}
  \end{equation*}
  where $\hat{w} = \hat{R}(w - c^*(w,z)) + \hat{Y}$

  \vspace{1em}

\end{frame}

%----- KEY ASSUMPTION ------------------------------------------------
\begin{frame}
  \frametitle{Key Assumption for Main Results}

  \begin{assumption}[2.2: Returns and Discounting]
    \begin{enumerate}
      \item For all $z \in Z$:
      \begin{itemize}
        \item $\E_z[u_c(\hat{Y}, \hat{Z})] < \infty$
        \item $\E_z[\hat{\beta}\hat{R} \cdot u_c(\hat{Y}, \hat{Z})] < \infty$
      \end{itemize}

      \item \structure{Spectral radius condition:} $r(K(1)) < 1$

      where the matrix $K(\theta)$ is defined by:
      \begin{equation*}
        K_{zz'}(\theta) = P(z,z') \int \beta(z,z',\varepsilon) R(z,z',\varepsilon)^{\theta} \pi(d\varepsilon)
      \end{equation*}
    \end{enumerate}
  \end{assumption}

  \begin{alertblock}{Interpretation}
    $r(K(1)) < 1$ ensures wealth doesn't explode in present value terms\\
    Generalizes the standard condition $\beta R < 1$
  \end{alertblock}
\end{frame}

%=====================================================================
\section{Main Results}

%----- THEOREM 2.1: EXISTENCE ----------------------------------------
\begin{frame}
  \frametitle{Existence and Uniqueness}

  \begin{theorem}[2.1: Existence and Uniqueness]

    If Assumptions 2.1 and 2.2 hold, then:
    \begin{enumerate}
      \item The time iteration operator $T:\mathcal{C}\to\mathcal{C}$ has a unique fixed point $c^*$.
      \item For any $c \in \mathcal{C}$:
      \begin{equation*}
        \rho(T^k c, c^*) \to 0 \quad \text{as } k\to\infty
      \end{equation*}
      where $\rho$ is a metric based on marginal utilities.
    \end{enumerate}
  \end{theorem}

  \vspace{0.5em}

  %% state MUC metric here 
\end{frame}

%----- THEOREM 3.2: ZERO MPC ----------------------------------------
\begin{frame}
  \frametitle{Zero Asymptotic MPCs}

  \begin{theorem}[3.2: Positive Transition Matrix]
    If every entry of $\bar{P}$ is strictly positive and $\Pr_{z,z'}(\hat{\beta}\hat{R} > 0) > 0$ for all $(z,z') \in Z^2$, then for all states with $i \neq 1$:
      \begin{equation}
        \lim_{w \to \infty} \frac{c^*(w, z_{ij})}{w} = 0
      \end{equation}
  \end{theorem}

  \vspace{0.5em}

  \begin{alertblock}{Key Insight (from more general result)}
    With fully mixing risk aversion Markov chain, zero asymptotic MPCs arise at all or almost all states.
  \end{alertblock}
\end{frame}

%----- INTUITION FOR MAIN RESULT ------------------------------------
\begin{frame}
  \frametitle{Economic Intuition}

  \begin{center}
    \large Why does the possibility of lower risk aversion drive saving?
  \end{center}

  \vspace{1em}

  \structure{The Mechanism}
  \begin{enumerate}
    \item Agent anticipates possible \alert{decrease} in risk aversion.
    \item Lower future risk aversion $\Rightarrow$ Higher future consumption desire.
    \item Creates \alert{perpetual precautionary saving motive}.
    \item Saving motive persists regardless of wealth level.
  \end{enumerate}

  \vspace{0.5em}
\end{frame}

\begin{frame}
  \frametitle{Comparison with existing theory}
  
  \structure{Key difference from existing theory:}
  
  \begin{itemize}

      \item Ma \& Toda (2021): Zero MPC is knife-edge case.
      \item \alert{This result}: Zero MPC is generic feature.
  \end{itemize}
  \structure{Result holds even with:}
    \begin{itemize}
      \item Constant returns ($R_t \equiv R$).
      \item No income risk.
      \item Constant discount factor.
  \end{itemize}
\end{frame}

%----- THEOREM 3.4: UPWARD TRANSITIONS ------------------------------
\begin{frame}
  \frametitle{Contrasting result: upward transitions}

  \begin{theorem}[3.4: Strictly Increasing Risk Aversion]
    If risk aversion strictly increases next period from state $i$ and $R$ is bounded below by $m > 0$:
    \begin{equation}
      \lim_{w \to \infty} \frac{c^*(w, z_{ij})}{w} = 1 \quad \text{for all } j
    \end{equation}
    (Consumption function is \alert{nonconcave}).
  \end{theorem}

  \vspace{0.5em}

\end{frame}

\begin{frame}
  \frametitle{Upward vs downward transitions}

  \begin{alertblock}{Key Contrast}
    \begin{itemize}
      \item \structure{Theorem 3.2}: Downward transitions $\Rightarrow$ $\bar{c} = 0$ (save everything).
      \item \structure{Theorem 3.4}: Upward transitions $\Rightarrow$ $\bar{c} = 1$ (consume everything).
    \end{itemize}
  \end{alertblock}

  \structure{Intuition:} When agents expect to become \alert{more} risk averse in the future, they consume aggressively today before their preferences change.
\end{frame}

%----- THEOREM 3.3: POSITIVE ASYMPTOTIC MPC -------------------------
\begin{frame}
  \frametitle{When MPCs can be positive}

  \vspace{0.5em}

  \begin{proposition}[3.1: No Downward Transitions Required]
    If $\bar{c}(z_{ij}) > 0$ for some $i,j$, then:
    \begin{equation}
      \bar{p}_{i1} = \cdots = \bar{p}_{i,i-1} = 0
    \end{equation}
    (Zero probability of moving to lower risk aversion).
  \end{proposition}
\end{frame}

%----- SUMMARY OF MAIN THEOREMS -------------------------------------
\begin{frame}
  \frametitle{Summary of main theorems}


  \structure{Main takeaway:}
  \begin{itemize}
    \item Zero MPCs are \alert{generic} when $\gamma$ can decrease.
    \item Strictly positive MPCs require \alert{restrictive} conditions disallowing downward transitions.
  \end{itemize}
\end{frame}

%----- COMPARISON WITH LITERATURE -----------------------------------
\begin{frame}
  \frametitle{Comparison with existing literature}
  \vspace{0.8em}

  \structure{Why this matters:}
  \begin{itemize}
    \item Single, intuitive mechanism, does not rely on growth rates.
    \item Robust theoretical result.
  \end{itemize}

  \structure{Why I want to know more:}
  \begin{itemize}
    \item Do we need limiting MPCs to be zero? What specific empirical fact are we trying to explain?
    \item Is this a story about risk or inter-temporal substitution?
    \item Overall, we are now left with lots of results with ad hoc growth conditions. 
    \item Does risk aversion dynamics pass the "Kath Day-Night Test"?
  \end{itemize}
\end{frame}

\begin{frame}
  \frametitle{Kath Day-Night test}

  \begin{center}
    Question: Why is your MPC limiting to zero?
    %
    \includegraphics[width=0.55\textwidth]{kath-kel-quiz-2.png}
  \end{center}

\end{frame}
%=====================================================================
\section{Implications}



\end{document}